% WIP-Stil LaTeX-Vorlage für (englischsprachige) Abschluss- und Studienarbeiten

\documentclass[
	11pt,								% Schriftgroesse
	DIV10,								% Koma-Script-Klassen: Groesse des bedruckbaren Bereichs
	a4paper,         					% Format
	oneside,							% Einseitiges Dokument
	headheight=20pt,					% Höhe Dokumentenkopf
	footheight=20pt,					% Höhe Dokumentenfuß
    parskip=full,						% Abstand zwischen Absaetzen (ganze Zeile)
    listof=totoc,						% Verzeichnisse im Inhaltsverzeichnis aufführen
	bibliography=totoc,					% Literaturverzeichnis im Inhaltsverzeichnis aufführen
	index=totoc,						% Index im Inhaltsverzeichnis aufführen
	%final
    %draft								% Status des Dokuments
]{scrartcl}

%%% Schrift %%%
\usepackage[utf8]{inputenc}				% Umlaute im Tex-Code erlauben
\usepackage[T1]{fontenc}
\usepackage[ngerman, english]{babel}	% Sprachpaket Deutsch
\usepackage[useregional]{datetime2}
\selectlanguage{english}
\usepackage{csquotes}					% Zitation im gleichen Stil 
\renewcommand{\baselinestretch}{1.50}	% Zeilenabstand 1.5
\normalsize
%\usepackage{uarial}						% Schriftart Arial, kann ausdokumentiert werden, wenn die Standardschrift benutzt werden soll
\renewcommand{\familydefault}{\sfdefault}
\usepackage{blindtext}					% zum Testen des Layouts
\usepackage{microtype}					% Verbesserter Randausgleich
\usepackage{color}
\usepackage{eurosym} 					% Eurozeichen
\usepackage{xcolor}
\usepackage{datetime}
\usepackage{dcolumn}
\usepackage{todonotes}

%%% Anpassung des Formats %%%
\RedeclareSectionCommand[
  beforeskip=-1sp,						% minimaler Abstand nach Überschrift, anschließend nicht einrücken
  afterskip=1sp]{section}
\RedeclareSectionCommand[
  beforeskip=-1sp,						% siehe oben 
  afterskip=1sp]{subsection}
\RedeclareSectionCommand[
 beforeskip=-1sp,						% siehe oben
  afterskip=1sp]{subsubsection}
\RedeclareSectionCommand[
  beforeskip=-1sp,						% siehe oben
  afterskip=1sp]{paragraph}
\RedeclareSectionCommand[
 beforeskip=-1sp,						% siehe oben
 afterskip=1sp]{subparagraph}

%%% Änderungen des Inhaltsverzeichnis %%%
\usepackage[titles]{tocloft}			% wird zur Änderung des toc verwendet

\renewcommand{\cftsecleader}			% führe Punkte im toc auch für sections ein, Punkte fetter als normal
	{\bfseries\cftdotfill{\cftdotsep}}
\renewcommand{\cftdotsep}{0.1}			% setze Punkteabstand enger	

%%% definiere Inhaltsübersicht %%%
% Achtung: In Vorlage auskommentiert %
\newcommand*\inhaltsuebersicht{%
\section*{Content Overview}				% Name 
\begingroup
\value{tocdepth}\shorttocdepth\relax
\makeatletter
\input{\jobname.toc}
\makeatother
\endgroup
}
\newcommand*{\shorttocdepth}{1}			% Tiefe der Inhaltsübersicht == 2

%%% Bilder %%%
\usepackage{graphicx}
\graphicspath{{./Bilder/}}				% Bilderverzeichnis
\usepackage{overpic} 					% Beschriftung in Abbildung platzieren
\usepackage{here}						% Bildplatzierung mit [H]

%%% Verschiedenes, Mathe-Funktionen, Layout %%%
\usepackage{float,caption}
\usepackage{amsmath,amsfonts}
\usepackage{mathtools}
\usepackage{amssymb}
\usepackage{exscale}
\usepackage[normalem]{ulem}
\usepackage{setspace}
\usepackage[a4paper,
    lmargin={2.5cm},
    rmargin={2.5cm},
    tmargin={2cm},
    bmargin={2cm}
    ]{geometry}
    
\addtolength{\footskip}{-0.5cm}			% Fussbereich 0.5cm höher, sodass die Seitennummierung höher ist

%%% Listen %%%
%\usepackage{enumitem}
\usepackage{paralist}					% Für kompakte Listen
\setlength{\pltopsep}{5pt}				% setzt den oberen Abstand der compactitem- und compactenum-Liste auf eine Zeilenbreite

%%% Tabellen %%%
\usepackage{tabularx}
\usepackage{booktabs}					% Platzverschwendung in Tabellen vermeiden
\usepackage{array}						% mehr Möglichkeiten der Darstellung

%%% Kopf und Fußzeile %%%
\usepackage[nouppercase,headsepline]{scrpage2}
\pagestyle{scrheadings}
\clearscrheadfoot
%\chead{\textit{TU Berlin, Fachgebiet Wirtschafts- und Infrastrukturpolitik (WIP)}}
	% Leerzeichen zwischen FN-Nummer und FN-Text
\deffootnote[]{1.5em}{1em}{\textsuperscript{\thefootnotemark}\enskip}
\cfoot{\normalsize{\emph{\thepage}}}

%%% URL's %%%
\usepackage[lowtilde]{url}				% bei Verwendung eines Tildezeichens wird es normal gesetzt
\urlstyle{same}
\usepackage{etoolbox}

%%% Zitation %%%
\usepackage[style=chicago-authordate,
			%style=authoryear,
            giveninits,					% Vornamen abkürzen
			maxbibnames=10,				% in der Bib werden 10 Autoren ausgegeben (bis zu 10)
            maxcitenames=2,				% max. 3 Autoren bei \cite
            backend=biber,
			doi=false,
			isbn=false,
			url=false,
            natbib=true,				% verwendet Natbib 
            sorting=nyt					
			]{biblatex}

\addbibresource{literatur.bib}
\usepackage{breakcites}

\DeclareNameAlias{author}{last-first}				

% *****************************************
%%% Anpassungen für englische Literatur %%%
% *****************************************

\DeclareFieldFormat[article]{title}{"#1\adddot"}				
\DeclareFieldFormat[inproceedings]{title}{"#1\addcomma"}				
%\DeclareFieldFormat[book]{title}\textit{{#1}\isdot}    			
\DeclareFieldFormat[techreport]{title}{"#1\adddot"}					
\DeclareFieldFormat[incollection]{title}{"#1\adddot"} 
\DeclareFieldFormat[online]{title}{#1\isdot}           		
\DeclareFieldFormat[misc]{title}{"#1\isdot"}					

\DeclareFieldFormat[inproceedings]{booktitle}{proceedings of #1}	

\AtEveryBibitem{\clearlist{language}}

\AtEveryBibitem{%
	\ifentrytype{online}
	{}
	{\clearfield{urlyear}\clearfield{urlmonth}\clearfield{urlday}}}

\AtEveryBibitem{%
	{\clearfield{month}\clearfield{day}}}

\DefineBibliographyStrings{english}{andothers = {et\,al\adddot}}

\DeclareFieldFormat[article]{number}{#1}					
\DeclareFieldFormat[article]{pages}{\space#1}

\usepackage{xpatch}
\xpretobibmacro{author}{\mkbibbold\bgroup}{}{}					% Autor fett
\xapptobibmacro{author}{\egroup}{}{}

\DeclareSourcemap{
	\maps{\map{
			\step[fieldsource=language, fieldset=langid, origfieldval, final]
			\step[fieldset=language, null]}}}

\usepackage[pdfborder={0 0 0},								% Rahmen in pdf nicht sichtbar
			breaklinks=true,
            %draft,											% alle Links abschalten
            ]{hyperref}

%%%%%%%%%%%%%%%%%%%%%%%%%%%%%%%%%%%%%%%%%%%%%%%%%%%%%%%%%%%%%%%%%%%%%%%%%%%%%%%%%%%%%%%%%%%%%%%
% Beginn des Dokument

% *************************************
%%% HIER DIE TITELSEITE BEARBEITEN %%%%
% *************************************

%%% Titelseite %%%
\newcommand{\autor}{Maximilian Eißler} 		% Name des Autors
\newcommand{\matriculation}{374881} 
\newcommand{\mailaddress}{eissler@campus.tu-berlin.de}

\newcommand{\betreuer}{Dr. Pao-Yu Oei} 		% Namen der Betreuer
\newcommand{\betreuerzwei}{Thorsten Burandt}
                      
                  
%\newcommand{\datum}{\selectlanguage{ngerman}\today}
\newcommand{\engdate}{\selectlanguage{english}\today}

%%% Glossar und Abkürzungsverzeichnis %%% 
\usepackage[acronym,
            nonumberlist,									% keine Seitenzahlen anführen
            toc,											% im Inhaltsverz. mit aufnehmen
            style=super,									% Einträge mit Abstand setzten 
            nopostdot,										% kein schließender Punkt
           	nogroupskip]									% kein Absatz zwischen Gruppen
           	{glossaries}									% muss nach \usepackage{hyperref} geladen werden, damit auch Einträge des Abkürzungsverzeichnisses verlinkt sind
\makeglossaries

%***********************************
%%% HIER DIE AKRONYME DEFINIEREN %%%
%***********************************

%%% exempl. Akronyme %%%
\newacronym[firstplural=renewable energy sources (RES)]{res}{RES}{renewable energy sources}
\newacronym{eeg}{EEG}{Erneuerbare-Energien-Gesetz}
\newacronym{tso}{TSO}{Transmission System Operator}
\newacronym{rmse}{RMSE}{root-mean squared error}
\newacronym{gams}{GAMS}{General Algebraic Modeling System}
\newacronym[firstplural=Energiewirtschaftsgesetzes (EnWG)]{enwg}{EnWG}{Energiewirtschaftsgesetz}
\newacronym{ihv}{i.H.v.}{in Höhe von}
\newacronym[firstplural=Millimetern (mm)]{mm}{mm}{Millimeter}
\newacronym[firstplural=Off\-shore-Wind\-parks (OWP)]{owp}{OWP}{Off\-shore-Wind\-park}
	% sort= ermöglicht korrekte Sortierung von Umlauten
\newacronym[sort=uebertr]{unb}{\"UNB}{Übertragungsnetzbetreiber}

% ***********************************************************************************
% Beginn des Dokuments
% ***********************************************************************************

\begin{document}\selectlanguage{english}
% ***********************************************************************************
% Automatische Zusammensetzung der Titelseite, nur aendern, falls noetig!
% ***********************************************************************************

	\thispagestyle{plain}
	\begin{titlepage}
		\vspace{0cm} 
		\begin{center}
			\includegraphics[width=2.5cm]{pictures/TU_Logo_kurz_4c_rot}\\
			\normalsize{Technische Universität Berlin}\\
			Fakultät VII Wirtschaft \& Management\\
			Fachgebiet Wirtschafts- und Infrastrukturpolitik (WIP)
		\end{center}
        \vfill
		%\vspace*{\fill}
		\begin{center}
			%\Large{\textbf{\textsc{\doctitle}}}\\
			\Large{\textbf{Bachelorarbeit}}\\
            \LARGE{\textbf{Title}}\\[2ex]
            \Large{\textbf{Subtitle}}
            
			\vfill
			
% "Ockerfarbener Kasten":

%{
%\setlength{\fboxrule}{0.2mm}
%\definecolor{ocker}{RGB}{196,188,150}
%			\selectlanguage{ngerman}\normalsize
%\fcolorbox{black}{ocker}{\begin{minipage}{\textwidth}

%\textbf{Dateinamen-Kern (mit Versions-Nr. 000):} 
%\url{vorlage_latex_v000}

%\textbf{Ablage-Server-Pfad (fixiert):} 
%\url{Z:\\lehre\\BEREICH_WIPOL_ETC\\veranstaltung___ewa_seminar\\_formatvorlagen\\LaTeX-Vorlagen\\Standard}

%\textbf{Aktueller Ablageort:} \url{\\\\afs\\tu-berlin.de\\units\\Fak_VII\\wip\\share\\lehre\\BEREICH_WIPOL_ETC\\veranstaltung___ewa_seminar\\_formatvorlagen\\LaTeX-Vorlagen\\Standard\\vorlage_latex_v001_vc_17-05-2017}

%\textbf{Datum:} \today \ \currenttime
%\end{minipage}}}

			\vfill
			\normalsize
			Author(s): \\
			\autor \, (\matriculation) - \mailaddress \\
            \vfill
			Supervisors:\\\betreuer	\\
			\betreuerzwei
			%\vspace*{\fill}
            \vfill
			Berlin, \engdate
		\end{center}
		
	\end{titlepage}
	\newpage
	\pagenumbering{roman}
	
% ***********************************************************************************
% Eidesstattliche Erklaerung, automatisch generiert
% ***********************************************************************************


\setcounter{page}{2}							% Seitenzahl == 2 (Titelseite wird mitgezählt)
% \chead{\textit{Eidesstattliche Erklärung}}
% \section*{Eidesstattliche Erklärung}
%	
%	Hiermit erkläre ich, \autor, an Eides statt, dass ich die vorliegende Arbeit selbstständig und nur unter Zuhilfenahme der ausgewiesenen Hilfsmittel angefertigt habe.\\
%	Sämtliche Stellen der Arbeit, die im Wortlaut oder dem Sinn nach anderen gedruckten oder im Internet verfügbaren Werken entnommen sind, wurden durch genaue Quellenangaben kenntlich gemacht.
%
%	

\chead{\textit{Statutory declaration}}
\section*{Statutory declaration}
	Hereby, I declare that I have developed and written this research completely by myself and that we have not used sources or means without declaration in the text. Any external thought, content, media, or literal quotation is explicitly marked and attributed to its respective owner or author. \\
	As of the date of submission, this piece of doument and its content have not been submitted anywhere else but to our supervisors.
	
	\bigskip
	
	Berlin, \engdate
	
	\bigskip
	\bigskip
		\begin{center}	
			\begin{minipage}[t]{0.3\textwidth}
				\rule[-0.2cm]{4.5cm}{0.5pt} \\
				\textsc{\autor}
			\end{minipage}
		\end{center}
	
	\newpage

% ****************************	
% Abstract und ggf. Zusammenfassung
% ****************************

	\chead{\textit{Abstract}}
	\section*{Abstract}
	
	Lorem ipsum dolor sit amet, consetetur sadipscing elitr, sed diam nonumy eirmod tempor invidunt
	ut labore et dolore magna aliquyam erat, sed diam voluptua. At vero eos et accusam et justo
	duo dolores et ea rebum. Stet clita kasd gubergren, no sea takimata sanctus est Lorem ipsum
	dolor sit amet. \\
	Lorem ipsum dolor sit amet, consetetur sadipscing elitr, sed diam nonumy eirmod tempor invidunt
	ut labore et dolore magna aliquyam erat, sed diam voluptua. At vero eos et accusam et justo
	duo dolores et ea rebum. Stet clita kasd gubergren, no sea takimata sanctus est Lorem ipsum
	dolor sit amet.
	
	\newpage	

% ***********************************************************************************
% Verzeichnisse
% ***********************************************************************************
	\selectlanguage{english}	
	
	\chead{\textit{Index}}
	\setcounter{tocdepth}{4} 					% Tiefe des Inhaltsverzeichnisses
    \setcounter{secnumdepth}{4}					% Tiefe der gezählten Überschriften
	
    \begingroup									% verringere Abstände in der Inhaltsübersicht
	\parskip=0pt	
    \endgroup
    \newpage
    \begin{onehalfspace}
    \chead{\textit{Contents}}
    \tableofcontents 							% Inhaltsverzeichnis
	\end{onehalfspace}
	\newpage
	\setkomafont{captionlabel}{\bfseries}
	\setkomafont{caption}{\bfseries}			% Bild- und Tabellenbeschriftung fett
		
	% Vor der Zahl steht Figure im Verzeichnis
	\renewcommand{\cftfigpresnum}{Figure }
	\settowidth{\cftfignumwidth}{Figure 5\quad}  
	\listoffigures								% Abbildungsverzeichnis
	%\newpage
		
	% Vor der Zahl steht Table im Verzeichnis
	\renewcommand{\cfttabpresnum}{Table }
	\settowidth{\cfttabnumwidth}{Table 150\quad}
	\newpage
	\listoftables								% Tabellenverzeichnis
	% \newpage
	%
	% %Glossary/List of Acronyms
	% \printglossary								% Glossar
	% \newpage
	% \printglossary[type=\acronymtype,			% Akronyme und Abkuerzungen
    %				title={List of Acronyms}]
    \newpage
	\pagenumbering{arabic}						% beginne mit "normaler" Nummerierung
	
% ***********************************************************************************
% CHEATSHEET
% Acronyms: \gls{tso}; \gls{gams}
%
%Chicago (author-date) citation style is used.\\
%In-text citation:
%\citet{abrell_integrating_2015} employed...
%
%Other forms of citation:
%.... (see \citealt{birge_introduction_2011})
%Please maintain a consistent table style (booktabs). For convenience, feel free to use a \href{https://www.tablesgenerator.com/}{TeX table generator}.
%\begin{table}[H]
%	\centering
%	\caption{An exemplary table}
%	\begin{tabular}{lll}
%		\hline
%		\textbf{First Column}		   & \textbf{Second Column}     & \textbf{Third Column}       \\ \hline
%		First Row                      & information     			& information     \\
%		Second Row                     & information     			& information     \\
%		Third Row					   & information     			& information     \\ \hline
%	\end{tabular}
%\end{table}
%\begin{flushleft}
%	\quad\quad\footnotesize{Source: Based on \citet{leuthold_large-scale_2012}.}
%\end{flushleft}
%\begin{figure}[H]
%	\centering
%	\includegraphics[width=\textwidth]{pictures/wind_turbine.png}
%	\caption{An exemplary figure}
%	\label{wind_turbine}
%	\flushleft\quad\quad\footnotesize{Source: Own illustration.}
%\end{figure}	
% ***********************************************************************************
	
% ***********************************************************************************
% Beginn des eigenstaendigen Teils
% ***********************************************************************************

\selectlanguage{english}


\chead{\textit{Introduction}}					% definiere Kopfzeile
%\include{files/Introduction}					% include statt input ermöglicht Arbeit mit \includeonly{} und fügt Seitenumbruch vor section ein

\section{Introduction}
In this document the topic, motivation and approach of my bachelor’s thesis shall be outlined. The aim is to give the reader an idea of what I try to accomplish with the choice of topic and methods, as well as a look at the first iteration of the mathematical model I am aiming to build.

\subsection{Motivation}
As the technological requirements for a decentralized energy system increasingly mature, the importance of understanding smaller units within the electricity system grows. There is a possibility of increasing reliance on small, partially autonomous grid units (microgrids) in the future. This calls for a better understanding of how such systems might operate. Especially the possible efficiency gains over a centralised system and the circumstances on which these efficiency gains might depend should be of interest to science and will be the focus of thesis.

\subsection{Research Question}
In my thesis I want to create a model of a small electricity grid with a single point of access to the main grid. The microgrid will contain 25 households of different types that will aim to reduce their total electricity costs. To attain this goal there will be two possible courses of action available: 
	\begin{enumerate}
	\item First, the option to invest in electricity generation and storage facilities, in this case only solar pv and battery storage. 
	\item Secondly, the possibility of unrestricted trade within the microgrid, meaning there will be no variable transaction costs, as they would usually occur in the form of fees and levies on electricity being transferred.		
	\end{enumerate}
	The aim of this approach is to determine the cost-saving potential in comparison to pure electricity consumption from the main grid in subject to the characteristics of the main parameter types to this model, which are:
	\begin{enumerate}
	\item The environmental conditions, primarily pertaining to the availability of renewable resources - in this case, solar irradiation - as well as temperature and changing seasonal energy demand.
	\item The 'behaviour' of the participants in this microgrid, primarily their willingness to shift or avoid loads depending on the current price of electricity within the microgrid or the main grid.
	\item The price of power from the main grid as a function of the time of day and  season, as well as the price of power generated by the households themselves.
	\end{enumerate}

\subsection{Methods}
	To model the outlined situation I want to use the computational modelling language 'Julia' and construct a linear optimisation model which can be solved with it. The goal is to minimise the cost of electricity for the households in the microgrid. It is notable at this point, that the non-consumption of electricity as well as the delay of consumption will be seen as a cost to the actor forgoing her demand. There will be an investment opportunity at the beginning of the examined timeframe of 20 years. The number of timeslices considered in the optimisation will be depending on performance of the model and can therefore not be determined yet. \\
	As a case study for this thesis I want to apply the model to the example of a community in Northrhine-Westfalia. With environmental variables set, I will then construct a number of scenarios with variations in electricity price and actor behaviour parameters. \\
	Another goal of this project is to make this a functional piece of research by reducing performance requirements as far as possible and creating an intuitive user interface that enables the user to 'play' with different scenarios. I prefer this approach over a few detailed and rigid scenarios because the future external factors are presently highly uncertain, which drives me to the conviction that an understanding of the dynamics involved in the examined system is of greater value than an exact solution to an unlikely scenario.
	
\subsection{Expected Results}
The result should be a tool that enables an intuitive understanding of the characteristics and the potentials of microgrids, even to non-economists. I intend an exemplary application of the model to the case of a Northrhine-Westfalian community. From this application I expect an insight into the efficiency gains achievable by microgrids in the German context. Also the required external factors for these efficiency gains to materialise should become evident. Under the right conditions I expect a double digit percentage drop in electricity costs compared to a purely consumption-based system. I hope to arrive at practical conclusions regarding present use cases for microgrids and required future regulation.
\newpage

\section{Literature Review}
In this section I shall attempt to give an overview over the current status quo regarding the optimizaton of microgrids and more especially the tools available to do so. I will pursue  this by answering a number of broad question, which I think are crucial regarding my subject. The procedure will be structured by utilizing a consistent and reproducible methodology described in detail below. In addition to scientific literature other sources will also be taken into account at my discrection if they are neccessary or helpful in answering a question.

\subsection{Key Questions}
This literature review will attempt to answer the following key questions in the context of my subject:
\begin{enumerate}
	\item What are the most eminent scientific standards for modelling a microgrid allocation and dispatch?
	\item What is usually within the scope of such a model (what types of generation assets, only electricity or also thermal energy and so on)? 
	\item What are the common methods used to determine some of the key variables required in such a model such as interest rates, CAPEX and OPEX of assets and so on?
	\item What are the most commonly used (commercial) tools for optimizing a microgrid? Are there any free or open source solutions? 
\end{enumerate} 

\subsection{Methodology}
To arrive at a dataset of scientific literature that is reproducible I use the methodology described in \todo{add reference to papers here}:

\begin{enumerate}
	\item Define a search string
	\item Choose scientific databases to which to apply that search string
	\item Due to the possibly large amount of papers brought up by this kind of search I am only considering the 100 most relevant papers from each database.
	\item Define keywords, which have to occur in the abstracts of the publications. All publications that lack a keyword are discarded.
	\item Define Inclusion as well as Exclusion criteria. A publication must satisfy all inclusion criteria as well as none of the exclusion criteria to be included in the literature review.
\end{enumerate}
Due to the nature of some of the questions I am trying to answer in my literature review it is additionally necessary to include further non-scientific sources at my discretion.
The search strings, used databases, as well as the dataset of literature at each step will be included in the appendix. \todo{include info and datasets in the appendix}

\subsection{Descriptive Analysis}
After filtering the original dataset of 275 unique publications in the way described in the last chapter, I arrive at a set of 61 publications. The publication year, as can be observed in Figure 1 is for most publications quite recently: 27 out of 61 papers were published 2017 and after. This could indicate a rising interest in the subject, but is probably at least partly due to the way the different search engines employed compute relevancy.  
\begin{figure}[H]
	\centering
	\includegraphics[width=\textwidth]{pictures/Figure_1.png}
	\caption{Results of an automated Descriptive Analysis}
	\label{descriptive_anaylsis_results}
	\flushleft\quad\quad\footnotesize{Source: Own illustration.}
\end{figure}	
Looking at the keywords used to describe the publication shows that, unsurprisingly, the most often occuring keywords are the ones used in conducting the search: 'linear programming', 'microgrids' and 'optimization'. 'distributed energy generation' and 'distributed energy resources'are mentioned 37 times, 'batteries' and 'energy storage' a total of 28 times, and 'renewable energy sources' and 'generators' a total of 16 times. This illustrates the focus on decentralized energy sources and storage, more specifically renewables and small scale combined heat and power generation that prevails throughout the literature.
\\
The publication chart is not very enlightening. This is due to the fact 28 out of 61 publications are conference papers which either appear under others because there is only one instance of that particular conference or unknown. From a more general point of view though almost all papers are published either by Elsevier or IEEE, with very few exceptions.


\subsection{Literature Overview}


\newpage
\newpage

\section{Appendix}
\subsection{Literature Review}
The used search String was:  [microgrid AND (optimization OR opimisation) AND linear programming]
\\
The used databases are ScienceDirect, IEEE Xplore and Google Scholar.
\\
The original dataset consistet of 300 publication of which 275 remained after duplicates where merged.
\\
The search string for the Abstract Keyword Search was [(microgrid OR micro-grid OR off-grid) AND (optimization OR optimisation OR optimise OR optimal OR optimally) AND (linear programming OR linear program OR mixed integer)
\\
After the abstract keyword search was conducted 106 publications remained.
\\
The inclusion criteria were:
\begin{enumerate}
	\item An optimization model is employed.
	\item There is some discussion about the design of the model.
	\item The objective of the model is the optimization of design or dispatch of a single microgrid.
	\item There is some sort of case study conducted.
\end{enumerate}
The exclusion criteria were:
\begin{enumerate}
	\item The publication is not in English language.
	\item The full text is not obtainable for this author with reasonable effort.
	\item The publication is a Work-In-Progress / Conference Paper version of a publication published in a journal and also included in this dataset.
	\item The mathematical model designed is non-linear.
	\item The mathematical model designed only considers a specific aspect of dispatch or design, not the entirety.
	\item The mathematical model is mostly focused on heat generation/distribution rather than electricity.
\end{enumerate}
After the filtering by inclusion and exclusion criteria 61 publications remained.

% weitere Dokumente einfügen mit den gleichen zwei Befehlen

%\nocite{*}									% gibt zum Testen des Literaturverzeichnisses alle Bibeinträge aus
\chead{\textit{References}}				    % siehe oben 
\renewcommand\refname{References}			% in Literaturverzeichnis umbenennen
\printbibliography
%[heading=bibintoc]
\end{document}
